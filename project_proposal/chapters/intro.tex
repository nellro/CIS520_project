\section{Introduction}
\label{sec:intro}
In this project we will compare classification methods for 
electrogram (ECG) arrhythmia discrimination. The set of methods we 
analyze include Deep Neural Network (DNN), Support Vector Machine 
(SVM), $k$-Nearest Neighbours (\knn) and Decision Tree.

After we complete the main part of our project,
% in addition
 we also plan to show  beneficial results of applying Principal 
 Component 
 Analysis (PCA) for speeding up the performance of multiple 
 computationally-intensive algorithms listed above. Though PCA is 
 usually used in unsupervised learning, we will show how it can be 
 applied to supervised learning algorithms such as DNN and SVM.  

The following two datasets will be used: 
\begin{itemize}
	\item \href{https://physionet.org/physiobank/database/mitdb/}{The 
	MIT-BIH Arrhythmia Database}~\cite{moody2001impact}. 
	It contains $48$ half-hour excerpts of two-channel ambulatory 
	Holter ECG recordings, obtained from $47$ subjects. The 
	recordings are digitized at $360$ samples per second. 
	There are approximately $110,000$ ECG
	beats in this database with $15$ different types of arrhythmia 
	including normal.
	The subjects were 25 men aged 32 to 89 years, and 22 women aged 
	23 to 89 years.
	\item EGM Database~\cite{jiang2016silico}. It consists
	of $1920$ EGM signals, equally divided between $960$ VTs and
	$960$ SVTs. The EGMs were generated by the heart
	model that has been validated for realism by 
	cardiologists~\cite{jiang2016silico}.
\end{itemize}
