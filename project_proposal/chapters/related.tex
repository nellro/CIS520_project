\section{Related work}
\label{sec:related}

Due to high importance of the problem, hundreds of methods for 
arrhythmia classification have been presented in the literature, 
%Data-driven approaches for discrimination became popular in 1990s, 
many of them use machine learning techniques. 
For instance, in~\cite{jun2018ecg} authors perform multi-class 
classification of 
ECG signals using convolutional neural network (CNN) to discriminate 
between different tachycardia types. 
%seven sophisticated types of tachycardias including 
%ventricular fibrillation, premature ventricular contractions, paced 
%beats, etc.
Authors in~\cite{assadi2015arrhythmias} present a $95.16\%$ accurate 
\knn{} classifier.
In 2012, other group of researchers reported a $99.34\%$ accurate 
multi-class arrhythmia classification algorithm using 
\textit{ensemble decision trees}. 
% 6 classes 
In \cite{mohanty2018ventricular} authors show that under specific 
conditions decision trees perform better than SVM classifiers.  







