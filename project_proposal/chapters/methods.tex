\section{Solution methods}
\label{sec:methods}

The set of methods we analyze implement and compare include: 
\begin{itemize}
	\item Deep Neural Network (DNN) (Al\"ena),
	\item Support Vector Machine (SVM) (Haotian),
	\item $k$-Nearest Neighbours (\knn)(Po-Yuan),
	\item Decision Tree (Haochen).
\end{itemize}

Related work review shows that decision trees can achieve high 
accuracy results in arrhythmia classification based on ECG signals. 
As the database for our project is also in ECG form, decision tree is 
a valid candidate for distinguishing fatal/non-fatal tachycardias. 
Compared with neural networks, decision trees have an advantage that 
it is easy to interpret. 
Unlike neural network, the transparency of decision tree gives us a 
better understanding of the intermediate processes.
 
\knn algorithm is known for its implementation simplicity and fast 
training time. 
It is also robust to noisy data. 
One potential difficulty is that $k-$NN is sensitive to 
irrelevant features. 
The biggest disadvantage of \knn is its expensive computation cost in 
the testing phase. 
Moreover, in case of high-dimensional data, the distance between the 
two data points becomes less meaningful and the accuracy may 
decrease~\cite{beyer1999nearest}. 

%PCA is a commonly used method to compress data and fasten 
%training process, and the data we will use is relatively large in 
%size (about 2MB per instance), so PCA could be useful, especially 
%when the model is not so simple, which requires a lot of calculation.

After accomplishing the main part of our project, we will focus on 
the possible modifications of the algorithms such as cost-sensitive 
decision tree, combining \knn and k-d tree, AdaBoost. 
% that might lead to computational improvements. We will 