\section{Solution methods}
\label{sec:methods}

The set of methods we analyze implement and compare include: 
\begin{itemize}
	\item Deep Neural Network (DNN) (Al\"ena),
	\item Decision Tree (Haochen),
	\item Support Vector Machine (SVM) (Haotian),
	\item $k$-Nearest Neighbours (\knn{})(Po-Yuan).	
\end{itemize}

Related work review shows that decision trees can achieve high 
accuracy results in arrhythmia classification based on ECG signals. 
As the database for our project is also in ECG form, decision tree is 
a valid candidate for distinguishing fatal/non-fatal tachycardias. 
Compared with neural networks, decision trees have an advantage that 
they are easy to interpret. 
Unlike neural network, the transparency of decision tree gives us a 
better understanding of the intermediate processes.
 
\knn{} algorithm is known for its implementation simplicity and fast 
training time. 
It is robust to noisy data, but sensitive to 
irrelevant features. 
The biggest disadvantage of \knn{} is its expensive computations in 
the testing phase. 
Moreover, in case of high-dimensional data, the distance between the 
two data points becomes less meaningful and the accuracy may 
decrease~\cite{beyer1999nearest}. 

%PCA is a commonly used method to compress data and fasten 
%training process, and the data we will use is relatively large in 
%size (about 2MB per instance), so PCA could be useful, especially 
%when the model is not so simple, which requires a lot of calculation.

After we complete the main part of our project,
we also plan to show  beneficial results of applying Principal 
Component 
Analysis (PCA) for speeding up the performance. Though PCA is 
usually used in unsupervised learning, we will show how it can be 
applied to supervised learning algorithms such as DNN and SVM.  
We will also focus on possible modifications of the algorithms such 
as cost-sensitive decision tree, combining \knn{}, and
AdaBoost. 
