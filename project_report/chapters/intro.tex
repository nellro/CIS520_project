\section{Introduction}
\label{sec:intro}

According to the World Health Organization (WHO), heart disease is 
the leading cause of death around the world.
More than $1000$ Americans died each day in $2016$ from heart 
attacks. 
Although a single arrhythmia heartbeat will not have a serious impact 
on life, continuous arrhythmia can result in fatal circumstances.
For instance, \textit{Ventricular Tachycardia} (VT) is a potentially 
fatal arrhythmia. 
% caused by abnormal electrical signals in the lower chambers of the 
%heart (ventricles).
On the other hand, \textit{Supraventricular Tachycardia} (SVT)
is not considered to be dangerous for most of people. 
% originates in the upper part of the heart (atria). 

In this project we compared classification methods for 
electrogram (EGM) arrhythmia discrimination. We consider two 
arrhythmia types: ventricular tachycardia (VT) and supraventricular 
tachycardia (SVT). Therefore, we viewed this problem as a
binary classification task to discriminate whether a patient has VT 
(fatal tachycardia) or SVT (non-fatal tachycardia).

The set of methods we considered, implemented and
analysed include Deep Neural Network (DNN), Support Vector Machine 
(SVM), $k$-Nearest Neighbours (\knn) and Decision Tree.


