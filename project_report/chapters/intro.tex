\section{Introduction}
\label{sec:intro}
In this project we compared classification methods for 
electrogram (EGM) arrhythmia discrimination. The set of methods we 
analysed include Deep Neural Network (DNN), Support Vector Machine 
(SVM), $k$-Nearest Neighbours (\knn) and Decision Tree.

\section{Data}
The following two datasets were used: 
\begin{itemize}
%	\item \href{https://physionet.org/physiobank/database/mitdb/}{The 
%	MIT-BIH Arrhythmia Database}~\cite{moody2001impact}. 
%	It contains $48$ half-hour two-channel ECG recordings, obtained 
%	from $47$ subjects. The 
%	recordings are digitized at $360$ samples per second. 
%	There are approximately $110,000$ ECG
%	beats. 
%	in this database with $15$ different types of arrhythmia 
%including normal.
%	The subjects were 25 men aged 32 to 89 years, and 22 women aged 
%	23 to 89 years.
	\item Cardiac model EGM database~\cite{jiang2016silico}. 
	% EGM - cardiac device incoming cardiac voltage signal
	It consists
	of $1920$ EGM signals, equally split into $960$ VTs and
	$960$ SVTs. The EGMs were generated by the heart
	model that has been validated for realism by 
	cardiologists~\cite{jiang2016silico}.
	\item Ann Arbor Electrogram Libraries~\cite{egm_data}. It 
	consists of 89 EGM recordings: 62 VT signals and 27 SVT signals. 
	Signals are of different length, varying from 10 seconds to 2 
	minutes recordings. 
\end{itemize}

Data pre-processing, following the methodology 
from~\cite{hajeb2018automated}, includes the following steps:
\begin{enumerate}
	\item Resample each signal with a sampling frequency of $300$Hz.
	\item Filter the signal with a fourth-order bandpass Butterworth 
	filter with low cut-off frequency of $0.4$Hz and high cut-off 
	frequency of $40$Hz to avoid baseline drift and reduce 
	high-frequency noise. 
	\item Segment the signal into $10$s data lengths.
	\item Extract 6 features from each signal (6 for each V signal 
	and 6 for each A signal, 12 in total): four features are derived 
	from image-based phase plot analysis 
	(see~\cite{hajeb2018automated}), one derived in the 
	frequency domain (number of frequencies which have higher 
	amplitude than the mean value), and the last feature is Shannon 
	Entropy~\cite{shannon1948mathematical} (The SE value is higher 
	for VT than for SVT).
\end{enumerate}
