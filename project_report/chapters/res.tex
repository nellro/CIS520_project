\section{Experimental results and Discussion}
\label{sec:res}

\begin{table}[]
	\begin{center}
	\begin{tabular}{|l|l|c|c|c|c|}
		\hline
		\multicolumn{2}{|l|}{}                                        
		                            & 
		\multicolumn{1}{l|}{\textbf{DNN}} & 
		\multicolumn{1}{l|}{\textbf{Decision tree}} & 
		\multicolumn{1}{l|}{\textbf{SVM}} & 
		\multicolumn{1}{l|}{\textbf{k-NN}} \\ \hline
		\multicolumn{2}{|l|}{\textbf{Accuracy 
		0-1}}                                               & 
		0.9219                            &     
		0.9313                           
		                  &     0.9437                
		&                    0.9437                \\ \hline
		\multicolumn{2}{|l|}{\textbf{Sensitivity}}                    
		                            & 
		0.9688                            &     
		0.9438                         
		                  &   0.9437                 
		&           0.90625                         \\ \hline
		\multicolumn{2}{|l|}{\textbf{Specificity}}                    
		                            & 
		0.875                            &     
		0.9188                      
		                  &  0.9437                   
		&            0.94375                        \\ \hline
		\multicolumn{2}{|l|}{\textbf{Precision}}                      
		                            &       
		0.8857                            &     
		0.9423                       
		                  &  0.9437                 
		&                     0.9432               \\ \hline
		\multicolumn{2}{|l|}{\textbf{F1}}                             
		                            & 
		0.9254                             &     
		0.9430                      
		                  &   0.9437               
		&                0.9028                    \\ \hline
		\multirow{4}{*}{\begin{tabular}[c]{@{}l@{}}Confusion\\ 
		matrix\end{tabular}} & \textbf{TP} 
		&    155/160            &   151/160                      
		                    &    
		                    151/160                               
		&                144/160                    \\ \cline{2-6} 
		& \textbf{TN} &     140/160                        
		&     147/160                                        
		&               
		      150/160              &    
		      151/160                                \\ 
		\cline{2-6} 
		& \textbf{FP} & 20/160                             
		&   13/160                                          
		&               
		    9/160                &          
		    16/160                          \\ 
		\cline{2-6} 
		& \textbf{FN} & 5/160                            
		&  9/160                                           
		&               
		   9/160                 &                  
		   9/160                  \\ 
		\hline
		\multicolumn{2}{|l|}{\textbf{Train time (overall)}} 
		&46.88s &0.2s &7.49s&-\\
		\hline
		\multicolumn{2}{|l|}{\textbf{Test time (per 
		patient)}}                                    &               
		       0.06ms             
		&       0.21ms                                      
		&               
		   1.58ms                 &               
		   0.97ms                     \\ 
		\hline
		\multicolumn{2}{|l|}{\textbf{Probability 
		estimation}}                                     & 
		+                                 &        -                   
		                  &  -                                
		&                                    \\ \hline
	\end{tabular}
\end{center}
\caption{Performance measures for all presented classification 
algorithms}
\label{tbl:res}
\end{table}


To compare algorithms we used the same dataset 
(see Section~\ref{sec:data}) for all algorithms: 1600 
training data points and 320 testing data points.
The performance measures are presented in a 
Table~\ref{tbl:res}.
All algorithms show high accuracy above $92\%$ and high sensitivity 
above $90\%$. 

Among all algorithms \knn{} shows highest accuracy, but the lowest 
sensitivity which is undesired in medical research. 
\knn{} algorithm can respond quickly when adding a new training 
point or taking out a training point from the set. However, \knn{} is 
sensitive to outliers and does not perform well on the imbalanced 
data. 
Moreover, \knn{} suffers from the curse of dimensionality: when the 
data dimension increases, the computation time grows rapidly. 
%Possible causes might result from 
%nonhomogeneous scale on features, which will affect distance 
%calculation. Also, outliers in the feature space may affect the 
%majority vote system even though we modified the weight calculation 
%method in KNN algorithm to decrease the effect of unstable feature 
%in 

Decision tree achieves the second best 
classification sensitivity after DNN while keeping test time low. 
The fact that this algorithm requires relatively small computations, 
and that it represents a set of simple classification rules, are the 
reasons why decision tree performs well with respect to test time. 
The disadvantage of decision tree approach is overfitting which 
occurs often. Moreover, when the input 
data has inconsistent sample size, its information gain tends to be 
characterized by the features which has more numerical values. 
However, those downsides can be avoided using Bagging trees (see 
Section~\ref{sec:tree}).

DNN achieves the best sensitivity result, which is very important in 
medical applications, but it also has the lowest accuracy value, 
which is also important for medical CPS.  
One also can see that DNN model makes very fast predictions in 
comparison with all other presented algorithms. This can be an 
implementation artefact, since Keras was developed with a focus on 
enabling fast experimentation and therefore, performs better in 
comparison with MATLAB libraries. 

SVM approach achieves comparable accuracy, sensitivity, and 
other performance 
measures with other approaches. 
However, when the dataset gets large, SVM requires more RAM 
and costs more time, which is a potential problem for medical 
applications. 



